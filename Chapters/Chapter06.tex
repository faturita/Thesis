\chapter{Event Related Potential: The P300 Wave}
\label{chapter:six}
\epigraph{This can be used to do this and that and that from Vidal paper!}{Vidal}

\section{Introduction}

The P300~\cite{Farwell1988,Knuth2006} is a positive deflection of the EEG signal which occurs around $300$ ms after the onset of a rare and deviant stimulus that the subject is expected to attend.  It is produced under the oddball paradigm~\cite{WolpawJonathanR2012} and it is consistent across different subjects. It has a lower amplitude  ($\pm 5 \mu V $) compared to basal EEG activity, reaching a Signal to Noise Ratio (SNR) of around $-15$ db estimated based on the amplitude of the P300 response signal divided by the standard deviation of the background EEG activity~\cite{Hu2010}.  This signal can be used to implement a speller application by means of a Speller Matrix~\cite{Farwell1988}. Fig.~\ref{fig:p300matrix} shows an example of the Speller Matrix used in the OpenVibe open source software~\cite{Renard2010}, where the flashes of rows and columns provide the deviant stimulus required to elicit this physiological response.   Each time a row or a column that contains the desired letter flashes, the corresponding synchronized EEG signal should also contain the P300 signature and by detecting it, the selected letter can be identified.


In response to this counting, a potential was elicited in the brain.  This response is kown as a P300 wave, as first reported by Sutton.  Detection of the responses an their timing in the measured signal made it possilbe to match rthe responses to one of the rows and one of the columns, and thus, the consen symbol cound be identified.

The flicker Effect (Neuro time series book) and their connection to SSVEP.  Verification of the dataset by means of SSVEP detection.

\section{Materias and Methods}

\subsection{Feature Extraction from Signal Plots} \label{Feature}

In this section, the signal preprocessing, the method for generating images from signal plots, the feature extraction procedure and the Speller Matrix identification are described.  Figure~\ref{fig:classification} shows a scheme of the entire process.

\subsubsection{Preprocessing Pipeline} \label{Pipeline}

The data obtained by the capturing device is digitalized and a multichannel EEG signal is constructed.

%A trial, as defined by the BCI2000 platform~\cite{Schalk2004}, is every attempt to select a letter from the speller. 

%It is composed of signal segments $S_{i}^l$ corresponding to $k_a$ repetitions of flashes of 6 rows and $k_a =10$ repetitions of flashes of 6 columns of the matrix, yielding 120 repetitions. 

The $6$ rows and $6$ columns of the Speller Matrix are intensified providing the visual stimulus.  The number of a row or column is a location. A sequence of twelve randomly permuted locations $l$ conform an intensification sequence. The whole set of twelve intensifications is repeated $k_a$ times.

%The multichannel EEG signal is processed on a channel by channel basis.   

\begin{itemize}
\item \textbf{Signal Enhancement}: This stage consists of the enhancement of the SNR of the P300 pattern above the level of basal EEG. The pipeline starts by applying a notch filter to the raw digital signal, a $4$th degree $10$ Hz lowpass Butterworth filter and finally a decimation with a Finite Impulse Response (FIR) filter of order $30$ from the original sampling frequency down to $16$ Hz \cite{Krusienski2006}.
\item \textbf{Artifact Removal}: For every complete sequence of $12$ intensifications of $6$ rows and $6$ columns, a basic artifact elimination procedure is implemented by removing the entire sequence when any signal deviates above/bellow $ \pm 70 \mu V $.
\item \textbf{Segmentation}: For each of the $12$ intensifications of one intensification sequence,  a segment $S_{i}^l$  of a window of $t_{max} $ seconds of the multichannel signal is extracted, starting from the stimulus onset, corresponding to each row/column intensification $l$ and to the intensification sequence $i$. As intensifications are permuted in a random order, the segments are rearranged corresponding to row flickering, labeled 1-6, whereas those corresponding to column flickering are labeled 7-12.  Two of these segments should contain the P300 ERP signature time-locked to the flashing stimulus, one for the row, and one for the column.
\item \textbf{Signal Averaging}: \label{Average}  The P300 ERP is deeply buried under basal EEG so the standard approach to identify it is by point-to-point averaging the time-locked stacked signal segments.  Hence the values which are not related to, and not time-locked to the onset of the stimulus are canceled out~\cite{Liang2008}.  

This last step determines the operation of any P300 Speller.  In order to obtain an improved signal in terms of its SNR,  repetitions of the sequence of row/column intensification are necessary.  And, at the same time, as long as more repetitions are needed, the ability to transfer information faster is diminished, so there is a trade-off that must be acutely determined.

The procedure to obtain the point-to-point averaged signal goes as follows:

\begin{enumerate}
\item \label{paso1}Highlight randomly the rows and columns from the matrix.  There is one row and one column that should match the letter selected by the subject.
\item  \label{paso2} Repeat step~\ref{paso1} $k_a$ times, obtaining the $1 \leq l \leq 12$ segments $S_1^l(n,c),\dots,S_{k_a}^l(n,c)$, of the EEG signal where the variables $1 \leq n \leq n_{max}$ and $1 \leq c \leq C$  correspond to sample points and channel, respectively. The parameter $C$ is the number of available EEG channels whereas $n_{max}=F_s \  t_{max}$ is the segment length and $F_s$ is the sampling frequency.  The parameter $k_a$ is the number of repetitions of intensifications and it is an input parameter of the algorithm.
\item \label{paso3} Compute the Ensemble Average by
\begin{equation}
x^l(n,c)= \frac{1}{k_a}\sum_{i=1}^{k_a}S_i^l(n,c) 
\label{averaging}
\end{equation}  
for $1 \leq n \leq n_{max}$ and for the channels $1 \leq c \leq C$.  This provide an averaged signal $x^l(n,c)$ for the twelve locations $ 1 \leq l \leq 12$.
\end{enumerate}
\end{itemize}


\subsubsection{Speller Matrix letter Identification}
\label{Classification}

\paragraph{P300 ERP Extraction}
Segments corresponding to row flickering are labeled 1-6, whereas those corresponding to column flickering are labeled 7-12.  The extraction process has the following steps:

\begin{itemize}
%\setcounter{enumi}{3}

\item \textbf{Step A:}\label{pasoa} First highlight rows and columns from the matrix in a random permutation order and obtain the Ensemble Average as detailed in steps~\ref{paso1}, \ref{paso2} and \ref{paso3} in Section \ref{Average}.
\item \textbf{Step B:}\label{paso4} Plot the signals $\tilde{x}^l(n,c)$,  $1 \leq n \leq n_{max}$, $1 \leq c \leq C $,  according Section~\ref{Plot} in order to generate the images $I^{(l,c)}$ for rows and columns $1 \leq l \leq 12$.

\item \textbf{Step C:} Obtain the descriptors $ \mathbf{d}^{(l,c)}$ for rows and columns from $I^{(l,c)}$  in accordance to the method described in Section~\ref{SIFT}. 

\end{itemize}

\paragraph{Calibration}

A trial, as defined by the BCI2000 platform~\cite{Schalk2004}, is every attempt to select just one letter from the speller.  A set of trials is used for calibration and once the calibration is complete it can be used to identify new letters from new trials.

During the calibration phase, two descriptors $ \mathbf{d}^{(l,c)}$ are extracted for each available channel, corresponding to the locations $l$ of a selection of one previously instructed letter from the set of calibration trials.  These descriptors are the P300 templates, grouped together in a template set called $ T^c $.   The set is constructed using the steps described in Section \ref{Average} and the steps A, B and C of the P300 ERP extraction process.

Additionally, the best performing channel, $bpc$ is identified based on the the channel where the best Character Recognition Rate is obtained.

\paragraph{Letter identification}

In order to identify the selected letter, the template set $T^{bpc}$ is used as a database.  Thus, new descriptors are computed and they are compared against the descriptors belonging to the calibration template set $T^{bpc}$.

\begin{itemize}

\item \textbf{Step D:} Match to the calibration template $T^{bpc}$ by computing  

\begin{equation}
\hat{row} = \arg \min_{l \in \{1,\dots,6\}} \sum_{q \in N_T(\mathbf{d}^{(l,bpc)})}^{} {\left\lVert q -  \mathbf{d}^{(l,bpc)} \right\rVert}  ^{2}
\label{eq:multiclassificationrow}
\end{equation}

\noindent and

\begin{equation}
\hat{col} = \arg \min_{l \in \{7,\dots,12\}} \sum_{q \in N_T(\mathbf{d}^{(l,bpc)})}^{} {\left\lVert q -  \mathbf{d}^{(l,bpc)} \right\rVert} ^{2}
\label{eq:multiclassificationcol}
\end{equation}

\noindent where $N_T(\mathbf{d}^{(l,bpc)})$  is defined as $N_T(\mathbf{d}^{(l,bpc)}) = \{\mathbf{d} \in T^{bpc} / $  is the k-nearest neighbor of $ \mathbf{d}^{(l,bpc)} \}$ for the best performing channel.  This set is obtained by sorting all the elements in $T^{bpc}$ based on distances between them and $\mathbf{d}^{(l,bpc)}$, choosing the $k$ with smaller values, with $k$ a parameter of the algorithm.  This procedure is based on the k-NBNN  algorithm~\cite{Boiman2008}.

\end{itemize}
By computing the aforementioned equations, the letter of the matrix can be determined from the intersection of the row $ \hat{row} $ and column $ \hat{col} $. 
Figure~\ref{fig:classification} shows a scheme of this process. 


\subsection{Experimental Protocol} \label{Protocol}

To verify the validity of the proposed framework and method, the public dataset 008-2014~\cite{Riccio2013} published on the BNCI-Horizon website~\cite{Brunner2014} by  IRCCS Fondazione Santa Lucia, is used. Additionally, an own dataset with the same experimental conditions is generated. Both of them are utilized to perform an offline BCI Simulation to decode the spelled words from the provided signals. 

The algorithm is implemented using  VLFeat~\cite{Vedaldi2010} Computer Vision libraries on MATLAB V2014a (Mathworks Inc., Natick, MA, USA). Furthermore, in order to enhance the impact of our paper and for a sake of reproducibility, the code of the algorithm has been made available at: https://bitbucket.org/itba/hist.

In the following sections the characteristics of the datasets and parameters of the identification algorithm are described. 


\subsubsection{P300 ALS Public Dataset} \label{ALSDataset}

The experimental protocol used to generate this dataset is explained in~\cite{Riccio2013} but can be summarized as follows:  8 subjects with confirmed diagnoses but on different stages of ALS disease, were recruited and accepted to perform the experiments. The Visual P300 detection task designed for this experiment consisted of spelling 7 words of 5 letters each, using the traditional P300 Speller Matrix~\cite{Farwell1988}. The flashing of rows and columns provide the deviant stimulus required to elicit this physiological response.  The first 3 words are used for calibration and the remaining 4 words, for testing with visual feedback.  A trial is every attempt to select a letter from the speller. It is composed of signal segments corresponding to $k_a =10$ repetitions of flashes of 6 rows and $k_a =10$ repetitions of flashes of 6 columns of the matrix, yielding 120 repetitions.  Flashing of a row or a column is performed for 0.125 s, following by a resting period (i.e. inter-stimulus interval) of the same length.  After 120 repetitions an inter-trial pause is included before resuming with the following letter.

The recorded dataset was sampled at 256 Hz and it consisted of a scalp multichannel EEG signal for electrode channels Fz, Cz, Pz, Oz, P3, P4, PO7 and PO8, identified according to the 10-20 International System,  for each one of the 8 subjects.   The recording device was a research-oriented digital EEG device (g.Mobilab, g.Tec, Austria) and the data acquisition and stimuli delivery were handled by the BCI2000 open source software~\cite{Schalk2004}.

In order to assess and verify the identification of the P300 response, subjects are instructed to perform a copy-spelling task. They have to fix their attention to successive letters for copying a previously determined set of words, in contrast to a free-running operation of the speller where each user decides on its own what letter to choose.

\subsubsection{P300 for healthy subjects}

We replicate the same experiment on healthy subjects using a wireless digital EEG device (g.Nautilus, g.Tec, Austria).  The experimental conditions are the same as those used for the previous dataset, as detailed in section~\ref{ALSDataset}.  The produced dataset is available in a public online repository~\cite{owndataset}.

Participants are recruited voluntarily and the experiment is conducted anonymously in accordance with the Declaration of Helsinki published by the World Health Organization.  No monetary compensation is handed out and all participants agree and sign a written informed consent.  This study is approved by the \textit{Departamento de Investigación y Doctorado, Instituto Tecnológico de Buenos Aires (ITBA)}.  All healthy subjects have normal or corrected-to-normal vision and no history of neurological disorders. The experiment is performed with 8 subjects, 6 males, 2 females, 6 right-handed, 2 left-handed, average age 29.00 years, standard deviation  11.56 years, range 20-56 years.

EEG data is collected in a single recording session. Participants are seated in a comfortable chair, with their vision aligned to a computer screen located one meter in front of them.  The handling and processing of the data and stimuli is conducted by the OpenVibe platform~\cite{Renard2010}. 

Gel-based active electrodes (g.LADYbird, g.Tec, Austria) are used on the same positions Fz, Cz, Pz, Oz, P3,P4, PO7 and PO8.  Reference is set to the right ear lobe and ground is preset as the AFz position.   Sampling frequency is slightly different, and is set to 250 Hz, which is the closest possible to the one used with the other dataset.

%Fz, Cz, P3, Pz, P4, PO7, PO8 and Oz. 

%8 gel-based active electrodes (g.LADYbird) + g.LADYbird (GND) + g.GAMMAearclip (REF) C3, Cz, C4, CPz, P3, Pz, P4, POz, GND: AFz, REF: right ear

\subsubsection{Parameters}

The patch size is $X_P = 12s \times 12s$ pixels, where $s$ is the scale of the local patch and it is an input parameter of the algorithm. The P300 event can have a span of $400$ ms and its amplitude can reach $ 10 \mu V $~\cite{Rao2013}.  Hence it is necessary to utilize a signal segment of size $t_{max} = 1$ second and a size patch $X_P$ that could capture an entire transient event. With this purpose in consideration, the $s$ value election is essential.

%necesitamos definir el valor de s en función de los parámetros de la señal, de modo tal que el parche cubra el evento completo.  
We propose the Equations~\ref{eq:mapping2} and~\ref{eq:mapping1} to compute the scale value in horizontal and vertical directions, respectively. 
\begin{equation}
s_x = \frac{ \gamma \;  \lambda \  F_s}{12}
\label{eq:mapping2}
\end{equation}

\begin{equation}
s_y= \frac{\gamma \; \Delta \mu V}{12} 
\label{eq:mapping1}
\end{equation}

\noindent where $ \lambda $ is the length in seconds covered by the patch, $ F_s $ is the sampling frequency of the EEG signal (downsampled to 16 Hz) and  $\Delta  \mu V $ corresponds to the amplitude in microvolts that can be covered by the height of the patch. The geometric structure of the patch forces a squared configuration, then we discerned that by using $ s =s_x =s_y = 3 $ and $ \gamma = 4 $,  the local patch and the descriptor can identify events of 9 $ \mu V $ of amplitude, with a span of $ \lambda = 0.56$ seconds.  This also determines that $ 1 $ pixel represents $ \frac{1}{\gamma}= \frac{1}{4} \mu V $ on the vertical direction and $\frac{1}{F_s \ \gamma}=\frac{1}{64}$ seconds on the horizontal direction. The keypoints  $\mathbf{p_k}$  are located at $ (x_{p_k}, y_{p_k} )= ( 0.55 F_s \ \gamma, z^l(c) )= (35,  z^l(c)) $ for the corresponding channel $c$ and location $l$ (see Equation~\ref{eq:zerolevel}).   In this way the whole transient event is captured. 
Figure~\ref{fig:patchgeometry} shows a patch of a signal plot covering the complete amplitude (vertical direction) and the complete span of the signal event (horizontal direction). 

Lastly, the number of channels $C$ is equal to $8$ for both datasets, and the number of intensification sequences $k_a$ is fixed to $10$.  The parameter $k$ used to construct the set $N_T(\mathbf{d}^{(l,c)})$ is assigned to $k=7$, which was found empirically to achieve better results.  In addition, the norm used on  Equations \ref{eq:multiclassificationrow} and \ref{eq:multiclassificationcol} is the cosine norm, and descriptors are normalized to $ \left[ -1, 1 \right] $.

\section{Results}

Table~\ref{tab:resultsals} shows the results of applying the Histogram of Gradient Orientations (HIST) algorithm to the subjects of the public dataset of ALS patients. The percentage of correctly spelled letters is calculated while performing an offline BCI Simulation.  From the seven words for each subject, the first three are used for calibration, and the remaining four are used for testing.  The best performing channel  $bpc$ is informed as well. The target ratio is $1:36$; hence theoretical chance level is $2.8\%$. It can be observed that the best performance of the letter identification method is reached in a dissimilar channel depending on the subject being studied.  Table~\ref{tab:resultsals} and~\ref{tab:resultsown} show for comparison the obtained performance rates using single-channel signals with the Support Vector Machine (SVM)~\cite{Scholkopf2001} classifier.  This method is configured to use a linear kernel.  The best performing channel, where the best letter identification rate was achieved, is also depicted.

%The spelled words are \textit{GATTO}, \textit{MENTE}, \textit{VIOLA} and \textit{REBUS}.

The Information Transfer Rate (ITR), or Bit Transfer Rate (BTR), in the case of reactive BCIs~\cite{WolpawJonathanR2012}  depends on the amount of signal averaging required to transmit a valid and robust selection.  Figure~\ref{fig:performance} shows the performance curves for varying intensification sequences for the subjects included in the dataset of ALS patients. It can be noticed that the percentage of correctly identified letters depends on the number of intensification sequences that are used to obtain the averaged signal.  Moreover, when the number of intensification sequences tend to 1, which corresponds to single-intensification character recognition, the performance is reduced. As mentioned before, the SNR of the P300 obtained from only one segment of the intensification sequence is very low and the shape of its P300 component is not very well defined.

In Table~\ref{tab:resultsown} the results obtained for 8 healthy subjects are shown.  It can be observed that the performance is above chance level. It was verified that HIST method has an improved performance at letter identification than SVM that process the signals on a channel by channel strategy (Wilcoxon signed-rank test, $p =  0.004$ for both datasets).

%In Tables~\ref{tab:resultsals} and~\ref{tab:resultsown} results for character recognition rates using single channel signals with the SVM~\cite{Scholkopf2001}  classification algorithm are also shown.    This algorithm was configured to use a linear kernel.  The best performing channel where the best letter identification rate was obtained is also depicted.

%The PE algorithm, which is also devised on a time-domain description of the waveform, was implemented according to \cite{Unakafova2013} and its parameters were adjusted as stated by \cite{Zanin2012}, with an \textit{order} of $2$ and a \textit{sliding window} of size $10$. 

Tables~\ref{tab:resultsalsswlda} and~\ref{tab:resultsownswlda} are presented in order to compare the performance of the HIST method versus a multichannel version of the Stepwise Linear Discriminant Analysis (SWLDA) and SVM classification algorithms for both datasets.  The feature was formed by concatenating all the channels~\cite{Krusienski2006}.  SWLDA is the methodology proposed by the ALS dataset's publisher. Since authors \cite{Riccio2013} did not report the Character Recognition Rate obtained for this dataset, we replicate their procedure and include the performance obtained with the SWLDA algorithm at letter identification.  It was verified for the dataset of ALS patients that it has similar performance  against other methods like SWLDA or SVM, which use a multichannel feature (Quade test with $p=0.55$) whereas for the dataset of healthy subjects significant differences were found (Quade test with $p=0.02$) where only the HIST method achieved a different performance than SVM (with multiple comparisons, significant difference of level $0.05$).

%It was verified for the dataset of ALS patients that it has similar performance  against other methods like SWLDA or SVM, which use a multichannel feature (Quade test with $p=0.55$) whereas for the dataset of healthy subjects significant differences where found (Quade test with $p=0.02$) where the HIST method achieved a better performance than SVM (with multiple comparisons, significant difference of level $0.05$).
 
%\subsection{Occipital Channels}

The P300 ERP  consists of two overlapping components: the P3a and P3b, the former with frontocentral distribution while the later stronger on centroparietal region~\cite{Polich2007}. Hence, the standard practice is to find the stronger response on the central channel Cz~\cite{Riccio2013}. However, \cite{Krusienski2006} show that the response may also arise in occipital regions.  We found that by analyzing only the waveforms, occipital channels PO8 and PO7 show higher performances for some subjects. 

%\subsection{Stability of the P300 shape}

As subjects have varying \textit{latencies} and \textit{amplitudes} of their P300 components, they also have a varying stability of the \textit{shape} of the generated ERP \cite{Nam2010}.  Figure~\ref{fig:p300templates} shows 10 sample P300 templates patches for patients 8 and 3 from the dataset of ALS patients. It can be discerned that in coincidence with the performance results, the P300 signature is more clear and consistent for subject 8 (A) while for subject 3 (B) the characteristic pattern is more difficult to perceive.

Additionally, the stability of the P300 component waveform has been extensively studied in patients with ALS \cite{SellersandEmanuelDonchin2006,TomohiroMadarame2008,Nijboer2009,Mak2012,McCane2015} where it was found that these patients have a stable P300 component, which were also sustained across different sessions.  In line with these results we do not find evidence of a difference in terms of the performance obtained by analyzing the waveforms (HIST) for the group of patients with ALS and the healthy group of volunteers (Mann-Whitney U Test, $p=0.46$). Particularly, the best performance is obtained for a subject from the ALS dataset for which, based on visual observation, the shape of they P300 component is consistently identified.

%\subsection{Descriptor Space and classification method}

It is important to remark that when applied to binary images obtained from signal plots, the feature extraction method described in Section \ref{SIFT} generates sparse descriptors.  Under this subspace we found that using the cosine metric yielded a significant performance improvement. On the other hand, the unary classification scheme based on the NBNN algorithm proved very beneficial for the P300 Speller Matrix.  This is due to the fact that this approach solves the unbalance dataset problem which is inherent to the oddball paradigm~\cite{Tibon2015}.  

%Using the same feature but with classification methods SVM, feed forward Neural Networks and SWLDA  common in BCI Research achieved a reduced performance.

\section{Conclusion}

%In this paper, a new unsupervised method to enhance evoked response by target stimuli in an oddball paradigm was presented. Only given the time indexes of rows/columns intensifications, the proposed algorithm estimates the main components of the P300 subspace by providing the best SNR. It was shown to efficiently improve the quality of the evoked responses by taking into account the signal and the noise, as opposed to principal component analysis, which only considers the signal. Using this method to enhance P300 subspace before the BCI classification task speeds up the BCI since less words are required to train the spatial filters and the linear classifier, given a certain percentage of good symbol prediction. Moreover, using this spatial enhancement significantly reduces the dimension of the feature vector used to predict words.


%For both datasets, the experimental protocol uses a very short inter-stimulus interval which has the potential to increase the ITR but at the same time it reduces the amplitude of the P300 response, hence it may be more difficult to detect it~\cite{Rao2013}.   It is known that ISI alters the P300 amplitude and may affect the chance to detect the ERP.

%In the case of the P300 response, the oddball paradigm requires that one of the stimuli be infrequent. Hence this forces the data to be unbalanced~\cite{Tibon2015}.  At the same time, the NBNN method suffers from biased classification on unbalanced classes~\cite{Fornoni2014}. %Para solucionar este problema, 

Among other applications of Brain Computer Interfaces, the goal of the discipline is to provide communication assistance to people affected by neuro-degenerative diseases, who are the most likely population to benefit from BCI systems and EEG processing and analysis.

In this work, a method to extract an objective metric from the waveform of the plots of EEG signals is presented.  Its usage to implement a valid P300-Based BCI Speller application is expounded.  Additionally, its validity is evaluated using a public dataset of ALS patients and an own dataset of healthy subjects. 

%The method works on a channel by channel basis; in this way the best performing channel can be identified and used it to reduce the number of required EEG electrodes, leading to the development of more ergonomic capturing device.

It was verified that this method has an improved performance at letter identification than other methods that process the signals on a channel by channel strategy, and it even has a comparable performance against other methods like SWLDA or SVM, which uses a multichannel feature.
Furthermore, this method has the advantage that shapes of waveforms can be analyzed in an objective way.  We observed that the shape of the P300 component is more stable in occipital channels, where the performance for identifying letters is higher.   We additionally verified that ALS P300 signatures are stable in comparison to those of healthy subjects.

%Further work should be conducted over larger samples to cross-check the validity of these results.

We believe that the use of descriptors based on histogram of gradient orientation, presented in this work, can also be utilized for deriving a shape metric in the space of the P300 signals which can complement other metrics based on time-domain as those defined by~\cite{Mak2012}. It is important to notice that the analysis of waveform shapes is usually performed in a qualitative approach based on visual inspection~\cite{SellersandEmanuelDonchin2006}, and a complementary methodology which offer a quantitative metric will be beneficial to these routinely analysis of the waveform of ERPs.

%and, based on this idea, we wanted to complement the methodology with a cuantitative and objective sight

The goal of this work is to answer the question if a P300 component could be solely determined by inspecting automatically their waveforms.  We conclude affirmatively, though two very important issues still remain:

First, the stability of the P300 in terms of its shape is crucial: the averaging procedure, montages, the signal to noise ratio and spatial filters all of them are non-physiological factors that affect the stability of the shape of the P300 ERP.  We tested a preliminary approach to assess if the morphological shape of the P300 of the averaged signal can be stabilized by applying different alignments of the stacked segments (see Figure~\ref{fig:classification}) and we verified that there is a better performance when a correct segment alignment is applied.  We applied Dynamic Time Warping (DTW)~\cite{Casarotto2005} to automate the alignment procedure but we were unable to find a substantial improvement.  Further work to study the stability of the shape of the P300 signature component needs to be addressed.

The second problem is the amplitude variation of the P300. We propose a solution by standardizing the signal, shown in Equation~\ref{eq:standarizedaverages}. It has the effect of normalizing the peak-to-peak amplitude, moderating its variation. It has also the advantage of reducing noise that was not reduced by the averaging procedure.   It is important to remark that the averaged signal variance depends on the number of segments used to compute it \cite{van2006signal}.  The standardizing process converts the signal to unit signal variance which makes it independent of the number $k_a$ of signals averaged.   Although this is initially an advantageous approach, the standardizing process reduces the amplitude of any significant P300 complex diminishing its automatic interpretation capability.

In our opinion, the best benefit of the presented method is that a closer collaboration of the field of BCI with physicians can be fostered \cite{Chavarriaga2017}, since this procedure intent to imitate human visual observation.  Automatic classification of patterns in EEG that are specifically identified by their shapes like K-Complex, Vertex Waves, Positive Occipital Sharp Transient~\cite{Hartman2005} are a prospect future work to be considered. We are currently working in unpublished material analyzing K-Complex components that could eventually provide  assistance to physicians to locate these EEG patterns, specially in long recording periods, frequent in sleep research~\cite{Michel2012}.  
Additionally, it can be used for artifact removal which is performed on many occasions by visually inspecting signals.  This is due to the fact that the descriptors are a direct representation of the shape of signal waveforms. In line with these applications,  it can be used to build a database~\cite{Chavarriaga2017} of quantitative representations of waveforms and improve atlases~\cite{Hartman2005}, which are currently based on qualitative descriptions of signal shapes.


%The spelled words are \textit{GATTO}, \textit{MENTE}, \textit{VIOLA} and \textit{REBUS}.

\begin{table}[htb]
\caption[Single Channel Character Recognition Rates for ALS patient's Dataset]{Character recognition rates for the public dataset of ALS patients using the Histogram of Gradient (HIST) calculated from  single-channel plots.  Performance rates using single-channel signals with the SVM classifier are shown for comparison.  The best performing channel $bpc$ for each method is visualized}
\centering
%% \tablesize{} %% You can specify the fontsize here, e.g.  \tablesize{\footnotesize}. If commented out \small will be used.
\begin{tabular}{c|cc|cc}
\toprule
\textbf{Participant}	&  $bpc$ 	&  HIST &  $bpc$	&  Single Channel SVM \\
\midrule
1     &     Cz   &   $35\%$    &  Cz   & $15\%$   \\
2     &     Fz   &   $85\%$      &  PO8   & $25\%$   \\
3     &     Cz   &   $25\%$    &  Fz   & $5\%$   \\
4     &     PO8 &   $55\%$   &  Oz   & $5\%$    \\
5     &     PO7 &   $40\%$    &  P3   & $25\%$   \\
6     &     PO7 &   $60\%$  &  PO8   & $20\%$    \\
7     &     PO8 &   $80\%$   &  Fz   & $30\%$     \\
8     &     PO7 &   $95\%$     &  PO7   & $85\%$ \\

%\bottomrule
\end{tabular}
\label{tab:resultsals}
\end{table}

%The spelled words are \textit{MANSO},\textit{CINCO},\textit{JUEGO} and \textit{QUESO}.

\begin{table}[htb]
\caption[Single Channel Character Recognition Rates for Healthy Subject's Dataset]{Character recognition rates for the own dataset of healthy subjects using the Histogram of Gradient (HIST) calculated from  single-channel plots.  Performance rates using single-channel signals with the SVM classifier are shown for comparison.  The best performing channel $bpc$ for each method is visualized.}
\centering
%% \tablesize{} %% You can specify the fontsize here, e.g.  \tablesize{\footnotesize}. If commented out \small will be used.
\begin{tabular}{c|cc|cc}
\toprule
\textbf{Participant}	&  $bpc$	&  HIST &  $bpc$	&  Single Channel SVM \\
\midrule
1     &     Oz   &   $40\%$  &  Cz   &  $10\%$    \\
2     &     PO7   &   $30\%$      &  Cz   & $5\%$   \\
3     &     P4   &   $40\%$    &  P3   & $10\%$    \\
4     &     P4 &   $45\%$    &  P4   & $35\%$     \\
5     &     P4 &   $60\%$  &  P3   & $10\%$     \\
6     &     Pz &   $50\%$ &  P4   & $25\%$     \\
7     &     PO7 &   $70\%$  &  P3   & $30\%$     \\
8     &     P4 &   $50\%$    &  PO7   & $10\%$    \\

%\bottomrule
\end{tabular}
\label{tab:resultsown}
\end{table}


\begin{table}[htb]
\caption[Character Recognition Rates for ALS patient's dataset]{Character recognition rates and the best performing channel $bpc$ for the public dataset of ALS patients using the Histogram of Gradient (HIST) (repeated here for comparison purposes). Performance rates obtained by SWLDA and SVM classification algorithms with a multichannel concatenated feature.}
\centering
%% \tablesize{} %% You can specify the fontsize here, e.g.  \tablesize{\footnotesize}. If commented out \small will be used.
\begin{tabular}{c|cc|c|c}
\toprule
%\textbf{Participant}	&  \textbf{BPC}	& \multicolumn{2}{c}{Character Recognition Rates}\\
%\cline{1-5} \\
\textbf{Participant}	&  $bpc$	&  HIST & Multichannel SWLDA & Multichannel SVM \\
                                    &  for HIST        &           &                                       &   \\
\midrule
1     &     Cz   &   $35\%$  & $45\%$  & $40\%$\\
2     &     Fz   &   $85\%$  & $30\%$   & $50\%$   \\
3     &     Cz   &   $25\%$  & $65\%$ & $55\%$   \\
4     &     PO8 &   $55\%$ & $40\%$  & $50\%$   \\
5     &     PO7 &   $40\%$ & $35\%$  & $45\%$   \\
6     &     PO7 &   $60\%$ &  $35\%$  & $70\%$   \\
7     &     PO8 &   $80\%$ & $60\%$   & $35\%$   \\
8     &     PO7 &   $95\%$  & $90\%$   & $95\%$  \\

%\bottomrule
\end{tabular}
\label{tab:resultsalsswlda}
\end{table}

\begin{table}[htb]
\caption[Character Recognition Rates for Healthy Subject's Dataset]{Character recognition rates and the best performing channel $bpc$ for the own dataset of healthy subjects using the Histogram of Gradient (HIST) (repeated here for comparison purposes).   Performance rates obtained by SWLDA and SVM classification algorithms with a multichannel concatenated feature.}
\centering
%% \tablesize{} %% You can specify the fontsize here, e.g.  \tablesize{\footnotesize}. If commented out \small will be used.
\begin{tabular}{c|cc|c|c}
\toprule
%\textbf{Participant}	&  \textbf{BPC}	& \multicolumn{2}{c}{Character Recognition Rates}\\
%\cline{1-5} \\
\textbf{Participant}	&  $bpc$ 	&  HIST & Multichannel SWLDA & Multichannel SVM  \\
                                    &  for HIST        &           &                                       &   \\
\midrule
1     &     Oz   &     $40\%$  &     $65\%$  &     $40\%$ \\
2     &     PO7   &     $30\%$ &   $15\%$  &     $10\%$ \\
3     &     P4   &     $40\%$ &     $50\%$  &     $25\%$ \\
4     &     P4   &     $45\%$ &     $40\%$  &     $20\%$ \\
5     &     P4   &      $60\%$ &    $30\%$  &     $20\%$ \\
6     &     Pz   &      $50\%$ &    $35\%$  &     $30\%$ \\
7     &     PO7   &      $70\%$ &  $25\%$  &     $30\%$ \\
8     &     P4   &      $50\%$ &    $35\%$  &     $20\%$ \\

%\bottomrule
\end{tabular}
\label{tab:resultsownswlda}
\end{table}


\begin{figure}[h!]
\centering
\includegraphics[width=15cm]{images/openvibep300matrix.png}
\caption[P300 Speller Matrix]{Example of the $6 \times 6$ Speller Matrix used in the study obtained from the OpenVibe software.  Rows and columns flash in random permutations.}
\label{fig:p300matrix}
\end{figure}

\begin{figure}[h!]
\centering
\includegraphics[width=15cm]{images/classificationgraph.pdf}
\caption[P300 Speller Matrix Letter Identification]{For each column and row, an averaged, standardized and scaled signal $\tilde{x}^l(n,c)$ is obtained from the segments $S_i^l$  corresponding to the $k_a$ intensification sequences with $ 1 \leq i \leq k_a $ and location $l$ varying between $1$ and $12$. From the averaged signal, the image $I^{(l,c)}$ of the signal plot is generated and each descriptor is computed.  By comparing each descriptor against the set of templates, the P300 ERP can be detected, and finally the desired letter from the matrix can be inferred.}
\label{fig:classification}
\end{figure}

\begin{figure}[h!]
\centering
\includegraphics[width=16cm]{images/gradients.png}\label{samplegradients}
\caption[Histogram of Gradient Orientations for ERP]{ (A) Example of a plot of the signal, a keypoint and the corresponding patch. (B) A scheme of the orientation's histogram computation.  Only the upper-left four blocks are visible.  The first eight orientations of the first block, are labeled from $1$ to $8$ clockwise. The orientation of the second block $ B_{1,2} $ is labeled from $9$ to $16$.  This labeling continues left-to-right, up-down until the eight orientations for all the sixteen blocks are assigned. They form the corresponding descriptor of $128$ coordinates.  The length of each arrow represent the value of the histogram on each direction for each block. (C) Vector field of oriented gradients.  Each pixel is assigned an orientation and magnitude calculated  using finite differences. }
\label{fig:sampledescriptor}
\end{figure}

\begin{figure}[h!]
\centering
\includegraphics[width=10cm]{images/patchgeometry.pdf}
\caption[Patch Geometry]{The scale of local patch is selected in order to capture the whole transient event.  The size of the patch is $X_p \times X_p$ pixels. The vertical size consists of $4$ blocks of size $3 s_y$ pixels which is high enough as to contain the signal $\Delta  \mu V $, the peak-to-peak amplitude of the transient event. The horizontal size includes $4$ blocks  of $3 s_x$ and covers the entire duration in seconds of the transient signal event, $ \lambda $.   }
\label{fig:patchgeometry}
\end{figure}

\begin{figure}[h!]
\centering
\includegraphics[width=10cm]{images/performance.eps}
\caption[P300 Performance Curves]{Performance curves for the eight subjects included in the dataset of ALS patients.  Three out of eight subjects achieved the necessary performance to implement a valid P300 speller.}
\label{fig:performance}
\end{figure}


\begin{figure}[h!]
\centering
\includegraphics[width=15cm]{images/subject.png}\label{subject8}
\caption[Sample P300 Patches]{Ten sample P300 template patches for subjects 8 (A) and 3 (B) of the ALS Dataset.  Downward deflection is positive polarity. }
\label{fig:p300templates}
\end{figure}

\begin{figure}[h!]
\centering
\includegraphics[width=15cm]{images/boxplots.png}
\caption[P300 Classification Boxplots]{Obtained boxplots for the given algorithms.}
\label{fig:boxplots}
\end{figure}
