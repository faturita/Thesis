\chapter{P300}

The P300~\cite{Farwell1988,Knuth2006} is a positive deflection of the EEG signal which occurs around $300$ ms after the onset of a rare and deviant stimulus that the subject is expected to attend.  It is produced under the oddball paradigm~\cite{WolpawJonathanR2012} and it is consistent across different subjects. It has a lower amplitude  ($\pm 5 \mu V $) compared to basal EEG activity, reaching a Signal to Noise Ratio (SNR) of around $-15$ db estimated based on the amplitude of the P300 response signal divided by the standard deviation of the background EEG activity~\cite{Hu2010}.  This signal can be used to implement a speller application by means of a Speller Matrix~\cite{Farwell1988}. Fig.~\ref{fig:p300matrix} shows an example of the Speller Matrix used in the OpenVibe open source software~\cite{Renard2010}, where the flashes of rows and columns provide the deviant stimulus required to elicit this physiological response.   Each time a row or a column that contains the desired letter flashes, the corresponding synchronized EEG signal should also contain the P300 signature and by detecting it, the selected letter can be identified.


In response to this counting, a potential was elicited in the brain.  This response is kown as a P300 wave, as first reported by Sutton.  Detection of the responses an their timing in the measured signal made it possilbe to match rthe responses to one of the rows and one of the columns, and thus, the consen symbol cound be identified.

The flicker Effect (Neuro time series book) SSVEP