\chapter{Visually Decoding Brain Signals}

Esta sección contiene el estado del arte del método

BCI + EEG

Cómo es el abordaje basado en Waveforms

Machine Learning

Shearlets

Scalar Space Theory


Diffusion Tensor Imagin

La idea es estructurar el paso a paso de como se puede ir usando el descriptor de gradientes de sift para 
mapear informacion.  Primero con una señal cruda, luego agregarle informacion extra, luego agregar ruido al azar, y finalmente empezar con información real de señales.  El tema luego se focaliza en EEG especificamente para BCI.


Describir la importancia de la impedencia (basado en el libro de Signal processing for neuro) con el paper que habla sobre EEG mas la pagina 144 del libro 2 de lotte.

Aca tambien pueden ir las referencias a la tesis de spinelli

Pattern Matching


Esta es la razon porque el metodo funciona ya que lo que termina detectando es de manera masiva
esas formas especificas que son las que le dan a las ondas alfa y mu sus nombres.

In human electrophysiology, oscillations with stereotyped nonsinusoidal shapes include the
sensorimotor mu rhythm, motor cortical beta oscillation, and cortical slow oscillations. The mu
rhythm oscillates at an alpha frequency (around 10 Hz) and was named because its waveform
shape resembles the Greek character m (Figure 1A). It is characterized by the fact that one
extremum (e.g., its peak) is consistently sharper than the other (e.g., its trough); it is also
described as an arch, comb, or wicket shape [4–10].
In addition to the sensorimotor mu rhythm, we have recently highlighted that motor cortical beta
oscillations also have striking nonsinusoidal features [11]. These beta oscillations manifest a
sawtooth shape in that their voltage either rapidly rises before more slowly falling off, or vice
versa (Figure 1B).