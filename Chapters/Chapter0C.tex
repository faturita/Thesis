\chapter{SIFT}

The history of Scale Space tracks back to Witkin 1983, where it was applied to time series.  He highlighted the Spatial Coincidential assumption.
Basically, the number of zero crossing of the first derivative is reduced with increasing scale.

\begin{story}[Biomimetic Applications]

\end{story}

This method is actually composed on two submethods: the first is the keypoint localization, while the second is the histogram of gradient orientations, which is the basis for this thesis.

bla bla bla


Aca también voy a mandar detalles de la implementacion.   como por ejemplo los detalles de como funciona vlfeat y las modificaciones.

\begin{itemize}
\item \url{https://bitbucket.org/itba/hist}
\item \url{https://github.com/faturita/BciVisualToolbox}
\item \url{https://github.com/faturita/vlfeat}
\item \url{https://github.com/faturita/GuessMe}
\end{itemize}

Datasets

\begin{itemize}
\item P300-Dataset \url{https://www.kaggle.com/rramele/p300samplingdataset}, Registered as public scientific resource in the public Database SciCrunch, RRID: SCR\_015977. 
\item P300 Template (routput.mat) and P300-null signal subject (P300-Subject-21.mat) at the CodeBase Repository \url{https://goo.gl/MzNNkn}.
\end{itemize}

Blogs

\begin{itemize}
\item The following blog was mantained during the development of this Thesis: \url{http://monostuff.logdown.com/}.
\end{itemize}

