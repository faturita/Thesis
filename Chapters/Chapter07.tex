\chapter{Conclusions}

BCI Security (IEEE Paper Life Science)



Sleep staging is one of the most important steps in sleep analysis. It is
a very time consuming task consisting of classifying all 30 second pieces
of an approximately eight hour recording into one of six sleep stages:
wakefulness, S1 (light sleep), S2, S3, S4 (deep sleep), REM (rapid eye
movement) sleep. A sleep recording is made with a minimum setting
of four channels: electro-encephalogram (EEG) from electrodes C3 and
C4 1, electro-myogram (EMG) and electro-oculogram (EOG). 

In order
to classify each 30 second segment of sleep according to the classical
[Rechtschaen  Kales 1968] (RK) rules, the human scorer looks for
defined patterns of waveforms in the EEG, for rapid eye movements in
the EOG and for EMG level. It is therefore a valuable goal to try and
automate this process and quite some work has already been done in
trying to replicate RK sleep staging with diverse automatic methods (see
[Hasan 1983] and [Penzel et al. 1991] for overviews). There is however a
considerable dissatisfaction within the sleep research community concerning
the very basis of RK sleep staging [Penzel et al. 1991]: RK is based on
a predened set of rules leaving much room for sub jective interpretation;