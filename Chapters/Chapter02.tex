\chapter{From signals to images}

%Contiene el método y el enfoque.

%Mental Chrometry and averaging 
%https://www.ncbi.nlm.nih.gov/pmc/articles/PMC548951/

This chapter describes the procedure to plot an image which represents the digital EEG signal.  This image is used to extract a feature which represents the waveform, the structure of the signal on a plot.  By analyzing this feature, we hypothesize that the underlying cognitive process can be detected and that it can be used to implement a brain-computer communication device.

\section{Electroencephalographic Plotting}

The plotting of the EEG is instrisically mixed with the nuisances of the electroencephalography itself.  Plotting proceed by using a chart recorded with a single pen (REF Jestico).   Voltages are represented in a vertical axis while time is represented in the horizontal axis, in a Cartesian arrangement. 

\begin{enumerate}
\item Sensitivity: also termed gain due the amplification procedure.  It was represented in $ \frac{mV}{mm}$.  In the digital form, it is $\frac{\mu V}{pixel}$.
\item Epoch/Paper speed: the time span that is represented in a single screen.  For paper strips it was used to be $10s$ (Figure FIG).  In the digital version is $ \frac{\lambda}{pixel}$.
\end{enumerate}

Logarithmic plotting, montage, calibration (cite Electroencphalograpy for empilepsy society)


\begin{story}[Neuroimaging]
With the advent of digital computers and the digital revolution, Plotting has become imaging.  Neuroimaging (CIte book Rodrigo Q Q) means mapping activity or structure to neuroanatomical regions (see figure anatomical structures).

There are currently three categories of neuroimaging: structural, functional, and chemical. 

\begin{itemize}
\item Structural Neuroimaging
Any technique that images structures of the brain. This would include CT (Computed Tomography), MRI (Magnetic Resonance Imaging), and DTI (Diffusion Tensor Imaging).

CT scanning is non-invasive uses x-rays to image tissue density. It is very rapid and can detect cerebral hemorrhaging in the early (acute) stage. It is most often, therefore, used for medical purposes.

Structural MRI is non-invasive and often provides better contrast resolution than CT with similar (and again, often better) spatial resolution. Unlike CT, structural MRI provides excellent tissue delineation, allowing users to visualize boundaries between grey and white matter in the brain, for example. Structural MRI is often used in neuroimaging to calculate volume of different brain regions or to define regions of brain damage or tumor.

DTI is non-invasive and can be done on most research MRI scanners. It involves using a special scanning and reconstruction sequence to image the flow (or, more specifically, constraints in the flow) of water through the brain. Because water flow is constrained by the axons (white matter) in the brain, it can be used to image large axonal connections between brain regions.

\item Functional Neuroimaging
Any technique that quantifies some metric of brain activity. This would include EEG (ElectroEncephaloGraphy), MEG (MagnetoEncephaloGraphy), fMRI (functional MRI), PET/SPECT (Positron Emission Tomography/Single Positron Emission Computed Tomography), NIRS (Near-InfraRed Spectroscopy), and, to a certain extent, TMS (Transcranial Magnetic Stimulation) and TDCS (Transcranial Direct Current Stimulation), along with several others.
\end{itemize}

\end{story}

\section{Signal to Image transformation}

The EEG signal may be represented by

\begin{equation}
\mathbf{x}(n) = x(n,c)
\label{eq:zerolevel}
\end{equation}

\noindent where $n$ are sample points digitalized at sampling frequency $F_s$.  This is a multichannel signal, for $c$ varying between $1$ and $C$.  Each one of this channels is assigned a name according to the 10-20 international system, and there are $C$ available channels.
 
To extract features from an image, first you need an image.  And this image needs to represent the underlying signal.  The straightforward way to do it, reproducing the traditional analog or digital EEG, is to draw a line on contrast background.  The line represents the voltage amplitude of the corresponding channel in relation to a zero-level $z(c)$, with a positive deflection going upwards (towards the zero value in the image coordinate system) and downwards for negative deflection (Figure \ref{fig:plot}).  If only two colors are used, this image is a black-and-white binary image.  In principle, color selection is completely arbitrary (white for the line, black for the background), but it has some implications in terms of the feature extraction procedure that we will describe later.

The value $z(c)$ is thus, the image vertical position where the signal's zero value has to be situated in order to fit the entire signal within the image for each channel c:

\begin{equation}
z(c) = \left \lfloor{ \frac{\max_{n} \tilde{x}(n,c)  - \min_{n} \tilde{x}(n,c) }{2} }\right \rfloor -   \left \lfloor{ \frac{\max_{n} \tilde{x}(n,c)  + \min_{n} \tilde{x}(n,c)}{ 2} }\right \rfloor
\label{eq:zerolevel}
\end{equation}

\noindent where the minimization and maximization are carried out for $n$ varying between ${1 \leq n\leq n_{max}}$, and $ \lfloor \cdot  \rfloor $ denote the rounding to the smaller nearest integer of the number.

In how many ways EEG signals can be mapped into 2D greyscale or binary images ?

\begin{equation}
I(x,y,n) = 0
\label{eq:standarizedaverages}
\end{equation}

which is a three dimensional surface.

\paragraph{Channel by Channel binary image }

The standard plotting, on a black-and-white image with lines representing voltage amplitude.

\begin{equation}
I(z_1,z_2) = \left\{ \begin{array}{rl}
255 & \text{if} \,  z_1 =  n; \; z_2 = x(n,c) + z(c) \\
0   & \mbox{otherwise}
\end{array}\right.
\label{eq:images}
\end{equation}

\paragraph{Channel by channel grey color image}

The voltage amplitudes are represented by grey scale colors, that could range between 0 and 255.  The function $\phi( \cdot )$ is a bounded linear mapping.

\begin{equation}
I(z_1,z_2) = \left\{ \begin{array}{rl}
\phi(x(n,c)) & \text{if} \,  z_1 = n; \; z_2 = z(c) \\
0   & \mbox{otherwise}
\end{array}\right.
\label{eq:images}
\end{equation}

\paragraph{Multichannel Full grey color image}

The image is grey-scale. Voltage amplitudes are represented by the pixel content and each channel is represented on the vertical axis.  The height of the signal is equal to the number of channels.   This is used in Neuroimaging plots of ERP events.

\begin{equation}
I(z_1,z_2) = \left\{ \begin{array}{rl} \phi(x(n,c))  & \text{if} \,  z_1 = n; \; z_2 = c \end{array}\right.
\label{eq:images}
\end{equation}


\paragraph{Multichannel Stationary Binary image}

The horizontal axis of the image is not time, but it is channels instead.   In this representation different contributions from different channels can be explored at the same time, but time dynamics is lost.  

\begin{equation}
I(z_1,z_2) = \left\{ \begin{array}{rl}
255 & \text{if} \,  z_1 = c; \; z_2 =  x(n,c) + z(n) \\
0   & \mbox{otherwise}
\end{array}\right.
\label{eq:images}
\end{equation}

\paragraph{Multichannel Stationary Grey-scale image}

This is a variant of the previous one, where the horizontal axis represent the channel.   In this form, the intensity of the contribution of each channel is represented by the grey-scale pixel value.  Combined with head models and forward projection solutions this is the approach used to represent scalp heatmaps (See figure 3, MNE).

\begin{equation}
I(z_1,z_2) = \left\{ \begin{array}{rl}
\phi(x(n,c)) & \text{if} \,  z_1 = c; \; z_2 =  z(n) \\
0   & \mbox{otherwise}
\end{array}\right.
\label{eq:images}
\end{equation}


\paragraph{Channel by channel full grey color image}

This is similar to a raster plot but the greyscale image representing voltages in pixel intensities is repeated $H$ times, which is the height of the image.  The selection of this value is arbitrary.

\begin{equation}
I(z_1,z_2) = \left\{ \begin{array}{rl} \phi(x(n,c_i))  & \text{if} \,  z_1 = n; \; z_2 = H \end{array}\right.
\label{eq:images}
\end{equation}

This last representation has the important property that $ LoG(I(x,0)) = LoG(x(n,c)) $.

\paragraph{Standardized plotting}

Averaged signal segments are standardized and scaled for $1 \leq n \leq n_{max}$ and $1 \leq c \leq C$ by 

\begin{equation}
\tilde{x}(n,c) = \left \lfloor{ \gamma \; \frac{( x(n,c) - \bar{x}(c)  )}{ \hat{\sigma}(c) } }\right \rfloor
\label{eq:standarizedaverages}
\end{equation}

\noindent where $\gamma > 0$ is an input parameter of the algorithm and  it is related to the image scale. In addition, $ x^l(n,c) $ is the point-to-point averaged multichannel EEG signal for the sample point $n$ and for channel $c$. Lastly, $$\bar{x}^l(c) =\frac{1}{n_{max}}\sum_{n=1}^{n_{max}}x^l(n,c)$$ and $$ \hat{\sigma}^l(c) = (\frac{1}{n_{max}-1}\sum_{n=1}^{n_{max}}(x^l(n,c)-\bar{x}^l(c))^2 )^{\frac{1}{2}}$$ are the mean and estimated standard deviation of $x^l(n,c), 1 \leq n \leq n_{max}$, for each channel $c$.

\paragraph{Autoscaled plotting}




\section{EEG Signal Plot}

A binary image $I^{(c)}$ is constructed according to

\begin{equation}
I^{(c)}(z_1,z_2) = \left\{ \begin{array}{rl}
255 & \text{if} \,  z_1 = \gamma \  n; \! z_2 = \tilde{x}(n,c) + z(c) \\
0   & \mbox{otherwise}
\end{array}\right.
\label{eq:images}
\end{equation}

\noindent with $255$ being white and representing the signal's voltage and $0$ for black which is the background contrast, conforming a black-and-white plot of the signal.  Pixel arguments $ (z_1,z_2) \in \mathbb{N} \times \mathbb{N}$ iterate over the width (based on the length of the signal segment) and height (based on the peak-to-peak amplitude) of the newly created image with $1 \leq n \leq n_{max}$ and $1 \leq c \leq C$.  

This equation produces a set of isolated pixels over the image.  To produce the plot $I^{(c)}$, the Bresenham \cite{Bresenham1965,Ramele2016} algorithm is used to interpolate straight lines between each pair of consecutive pixels.

