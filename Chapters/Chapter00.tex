\chapter*{Introduction}

During the last years of the 20th century an emerging idea took form and shaped one of the oldest dreams of mankind.

In terms of Schwartz Laboratory, the bandwidth of communication based on HCI devices seems very low (cite)

As mentioned in the most cited paper of the discipline, there are three motives behind BCI: the first is the Aging Societies: estimated for 2025, 800 millions people will be over 65 years old, and $2/3$ of them on developing countries.  This may lead to an increased tendency to develop diseases that affect motor pathways, that will require some for of assistance from technology. At the same time, science has provided the understanding that an Active Lifestyle is necessary to maintain body and mind health.  It is known that the ability to walk independently is a key indicator of psychological and physical health.   Last, but not least, the digital world demands more methods of interactions.  Digital society demands more mechanisms to interpret our surrounding  world and to translate our intentions through digital gadgets.

Objectives of this work

This thesis tries to unravel the following question:  it is possible to analyze and discriminate brainwaves signals, particularly Electroencephalographic signals, out of the automatic processing of the shape of waveforms using the Histogram of oriented gradients ?

To do that, I ask the reader to join me following this path:  (1) we will give details of what is Brain Computer Interfaces and the particularities of the first window of electric mind: the EEG.  (2)  we will cover also the state of the art in the methods that explore the waveform automatically.  The following section will give an overview about how a digital plotting is actually performed. The following section is the core of this theses and we will describe the Histogram of Gradient Orientations and how it can be used to process one dimensional signals.
Next, results and experimental procedures will be described, particularly Alpha Waves (4.3), Motor Imagery (5.2), P300 (6) and SSVEP.  Future Work and Conclussions will be addressed in 66.5.  Additionally, appendixes provide extra more information which is not extrictly related to this thesis but which provides information regarding the state-of-the-art of this discipline in Argentina, and also provides more inforamtion about the SIFT method and the theory behind the Histogram of Gradient Orientations of Signal Plots.


