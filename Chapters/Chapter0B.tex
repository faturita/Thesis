\chapter{Negative Results}

This is a list of non-curated techniques that we have tested.

\begin{itemize}
\setlength\itemsep{-0.8em}
\item Signal plotting: reducing edge threshold and low contrast threshold.
\item Siftmatch algorithm on Alpha Waves.
\item Bag of Words
\item Clustering of descriptors using kmeans, kmedoids
\item Extended Multi Siftmatching.  Scoring of descriptors based on the matching performance.
\item ICA + SIFT
\item PCA + SIFT
\tiem Applying ICA to SIFT Descriptors.
\item Apllying PCA to SIFT Descriptors.
\item Siftmatch on Bereitschaftspotentials.
\item Using SURF instead of SIFT
\item Using GLOH instead of SIFT
\item Using HOG instead of SIFT
\item Using PCA-SIFT instead of SIFT
\item Using dense SIFT instead of SIFT
\item Using Multiscale Harris Feature Detector instead of SIFT
\item Using Local Binary Patterns instead of SIFT
\item Siftmatching
\item Applying NNDR: Near neighbor distance ratio
\item Classifying through LDA/QDA/SVM
\item SIFT Descriptors normalization
\item Signal z-score
\item Signal re-referencing
\item Plotting: Logarithmic plotting, different scales and sizes.
\item Unsupervised Clustering: Cloud of Descriptors and Bag of Descriptors
\item Increased SIFT vector with one extra dimension (time).
\item Classification using voting scheme on clusters:  cluster homogenization.
\item Density Clustering: DBSCAN and OPTICS
\item Supersymmetric Clustering (Spike Sorting)
\teim Clustering Classification using Powel Method.
\item Unsupervised Learning: clustering, labeling and voting.
\item DBSCAN Cluster + Clustering Labeling based on Voting + kNN + Voting (it did not generalize well)
\item Multidimensional Scaling SIFT Descriptors
\item Using artificial EEG (Jensen-Ret Model)
\item Raster SIFT Descriptors
\item Bypass SIFT Detector: Manual keypoint localization
\item DBSCAN Radio Problem
\item DBSCAN + SIFT on BP
\item Using OPTICS to solve DBSCAN radio problem
\item Voting using weighted descriptors.
\item Usage of Ensemble Classifiers: stacked.
\item Clustering using DENCLUE and CLIQUE.
\item DBSCAN + SIFT on P300/SSVEP and BP
\item DBSCAN + SIFT on Motor Imagery (BCI Competition IV 2a)
\item Classification using NBNN algorithm on Alpha Waves and Motor Imagery (It worked!)
\item Channel Selection using averaging on all channels.
\item SIFT + NBNN on P300 (BCI Competition II 2003, 2b)
\item Preprocessing signals: different forms of decimation and downsampling.
\item SIFT + NBNN on P300 (008-2014)
\item Signal Averaging (DTW, Savitzky–Golay, boxcar, interleaved, MAD, median, outlier robust sigmoid transform (Fulcher 2013))
\item Rebalance P300 dataset (Striking a balance..., Tibon2015)
\item SIFT + NBNN on P300 (003-2015)
\item Signal averaging, testing on single trial
\item NBNN + NNDR
\item Weighted NBNN
\item SIFT + NBNN using only one keypoint/patch/descritor for image
\item SIFT: free/fixing octave
\item SIFT: choosing different octaves
\item SIFT: force specific orientation
\item SIFT: enable/disable first octave smoothing
\item SIFT: enable/disable Gaussian weighting.
\item SIFT: Rectangular patch
\item Preprocessing: elimination of segments with big variances.
\item Adaptative NBNN: weighting descriptors
\item Plot descriptors with radar plots.
\item Averaging across trials and channels.
\item Segments windowing (Hamming).
\item Baseline removal
\item Keypoint location: locate it at half the image size (it works!)
\item Signal Averaging (median)
\item Decimate the signal and the image while calculating SIFT octaves.
\item Image scaling on amplitude and time
\item Change segment size.
\item Standardize the signal using zscore to fix variance issue while averaging.
\item Averaging of signals after ICA and PCA filtering
\item Classifying P300 selecting templates by hand.
\item Classification using Kohonen Neural Net.
\item Different scales for the descriptor
\item Using SMOTE to rebalance the dataset
\item Classify the hit signal against the 6 others for row and 6 others for column, inverting the NBNN classification scheme (it works!)
\item Changing to COSINE distance
\item Keypoint location: varying horizontal locations, peak-picking and choosing barycenter.
\item Changing to Hellinger distance
\item Signal resampling
\item NBNN: extended to K=7 neighbors (works better)
\item Classification using SWLDA/SVM/NN
\item Multiclass NBNN for P300
\item SIFT + NBNN on KComplex
\item SIFT + NBNN on SSVEP
\end{itemize}


