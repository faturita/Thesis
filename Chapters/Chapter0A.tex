\chapter{BCI en Argentina}
\label{chapter:ten}

El propósito de este apéndice es ofrecer información del estado de esta disciplina en Argentina.  La inevitable omisión de trabajos o grupos específicos de ninguna manera ha sido adrede, y se solicita las pertinentes disculpas.  Este relevamiento fue realizado durante el transcurso del desarrollo de esta tesis, y probablemente tenga una visión desde el mundo de las Ingenierías, lo cual es un sesgado abordaje en esta tremendamente interdisciplinaria disciplina. \\

Los pioneros en Argentina son los trabajos en la Universidad de La Plata, y los trabajos de la UNER.  Estos últimos desarrolladron en el 2006 las primeras Jornadas Argentinas sobre Interfaces Cerebro-Computadora y las replicaron en el JAICC 2009. \\

\begin{itemize}
\item UNER, Faculta de Ingeniería, LIRINS,(Oro Verde) Bioingeniería  Dr. Gerardo Gentilleti
\begin{itemize}
\item \url{http://cortex.loria.fr/Projects/STIC-AmSud-BCI},
\item \url{http://www.bioingenieria.edu.ar/postgrado/index.php?option=com_content&view=category&id=72&Itemid=61}
\item Otros investigadores: Guerenstein, Pablo; Carolina B. Tabernig (BCI-FES system for neuro-rehabilitation of stroke patients LINK)
\end{itemize}
\item FRN: Fundación Rosarina de Neuro-Rehabilitación, Medicina y Rehabilitación, Dr. Carlos Ballario
\begin{itemize}
\item BCI-FES
\item Stroke Neurorehabilitation
\item Trabajan con la Empresa Interactive Dynamics. 
\end{itemize}
\item UBA, Facultad de Ingeniería, Laboratorio de Sergio Lew (\url{http://www.fi.uba.ar/es/node/1442}) , "Instituto de Ingeniería Biomédicas" / Dr. Sergio Lew
\begin{itemize}
\item BCI Invasivo principalmente (trabajan cerca de Zanutto).
\end{itemize}
\item UBA, Ingeniería, Laboratorio de Sistemas Inteligentes Dr. Jorge Ierache \url{http://laboratorios.fi.uba.ar/lsi/}.
\begin{itemize}
\item Control de robots por bioseñales, detección de emociones.
\end{itemize}
\item UBA, Exactas, \url{https://liaa.dc.uba.ar/} Applied Artificial Intelligence Lab Dr. Agustín Gravano / Dr. Diego Fernandez Slezak
\begin{itemize}
\item Tesis de grado Arneodo.
\item Otros investigadores: Alejandro Sabatini
\end{itemize}
\item INAUT, Instituto Nacional de Automática, San Juan, / Dr. Carlos Soria, Dr. Eugenio Orosco
BCI Robótica (BCI híbridos, robótica asistiva)
Trabajan con Teodiano Freire Bastos en Brasil
\url{www.ncbi.nlm.nih.gov}
Otros investigadores: Mst. Ing. Fernando Auat Cheeín
E-mail: fauat@inaut.unsj.edu.ar
\item Instituto Argentino de Matemáticas Alberto Calderon / Bioing. Sergio Liberczuk, Dr. Bruno Cernuschi Frías
Matemáticas y modelado del problema inverso.
\item ITBA, / CiC del Dr Juan Santos, http://www.itba.edu.ar/es/id/centros/cic-centro-de-inteligencia-computacional
Proyecto Doctorado Robótica Asistiva BCI Neurorehabilitación, Rodrigo Ramele
\url{http://www.unsam.edu.ar/tss/controlar-maquinas-con-el-pensamiento/978-3-319-13117-7_142}
\item UNC, Universidad Nacional de Cordoba
Trabajo Final de Ingeniería: \url{http://www.electronicosonline.com/2013/07/08/crean-jovenes-argentinos-interface-cerebral-para-discapacitados/}
Carrera de Ingeniería Biomédica: Ing. Diego Beltramone
\item UNLP, LEICI / Dr. Enrique Spinelli (\url{http://www.ing.unlp.edu.ar/leici/esp/pspinelli.html})
Electrónica.
Tesis de Grado de García Pablo: \url{http://sedici.unlp.edu.ar/handle/10915/3800631605}
Tesis de Maestría de Andrea Noelia Bermudez Cicchino 31605
Cesar Caiafa (trabajó con Cichocki) \url{http://ccaiafa.wixsite.com/cesar}
\item Universidad Nacional de Tucuman, Instituto Superior de Investigaciones Biológicas (INSIBIO)
\url{www.lamein.org}
Investigación sobre alternativas de codificación neural de los sistemas sensoriales.
Investigadores responsables: Dr. Carmelo Felice, Mst. Ing. Fernando Farfán
E-mail: cfelice@herrera.unt.edu.ar, ffarfan@herrera.unt.edu.ar
\item Laboratorio de Investigación y Desarrollo en Nuevas Tecnologías (LIDeNTec) - ANSES
Desarrollo de BCI
Investigadores responsables: Dr. Mario Mastriani
E-mail: mmastri@gmail.com
\item INECO: Cercanía con BCI pero no parece ser el foco de lo que hacen.
Eugenia Hesse \href{https://neuro.org.ar/sites/neuro.org.ar/files/Hesse-2015-Early%20detection%20of%20intentional%20harm.pdf}
Agustín Ibañez
\item  IBCN Silvia Kochen
\url{http://www.ibcn.fmed.uba.ar/200_grupos-lab-epilepsia-kochen.html}
\item Instituto Ferrero de Neurología y Sueño. Fundación Argentina de Estudio del Cerebro.
Publicaron un excelente libro en castellano de \textit{Análisis Computado de EEG}.

\end{itemize}