\chapter{BCI en Argentina}
\label{chapter:ten}

El propósito de este apéndice es ofrecer información del estado de esta disciplina en Argentina.  La inevitable omisión de trabajos o grupos específicos de ninguna manera ha sido adrede, y se solicita las pertinentes disculpas.  Este relevamiento fue realizado durante el transcurso del desarrollo de esta tesis, y probablemente tenga una visión sesgada desde el mundo de las ingenierías, lo cual es un abordaje equivocado en esta tremendamente interdisciplinaria disciplina. \\

Los pioneros en Argentina son los trabajos en la Universidad de La Plata, y los trabajos de la UNER.  Estos últimos organizaron en el 2006 las primeras Jornadas Argentinas sobre Interfaces Cerebro-Computadora JAICC, que replicaron en el año 2009. \\

\begin{itemize}
\item UNER, Faculta de Ingeniería, LIRINS,(Oro Verde) Bioingeniería  Dr. Gerardo Gentilleti
\begin{itemize}
\item \url{http://cortex.loria.fr/Projects/STIC-AmSud-BCI},
\item \url{http://www.bioingenieria.edu.ar/postgrado/index.php?option=com_content&view=category&id=72&Itemid=61}
\item Otros investigadores: Guerenstein, Pablo; Carolina B. Tabernig (\textit{BCI-FES system for neuro-rehabilitation of stroke patients})
\end{itemize}
\item FRN: Fundación Rosarina de Neuro-Rehabilitación, Medicina y Rehabilitación, Dr. Carlos Ballario
\begin{itemize}
\item BCI-FES
\item Stroke Neurorehabilitation
\item Trabajan con la Empresa Interactive Dynamics. 
\end{itemize}
\item UBA, Facultad de Ingeniería, Laboratorio de Sergio Lew (\url{http://www.fi.uba.ar/es/node/1442}) , "Instituto de Ingeniería Biomédicas" / Dr. Sergio Lew
\begin{itemize}
\item BCI Invasivo principalmente (trabajan cerca de Zanutto).
\end{itemize}
\item UBA, Ingeniería, Laboratorio de Sistemas Inteligentes Dr. Jorge Ierache \url{http://laboratorios.fi.uba.ar/lsi/}.
\begin{itemize}
\item Control de robots por bioseñales, detección de emociones.
\end{itemize}
\item UBA, Exactas, \url{https://liaa.dc.uba.ar/} Applied Artificial Intelligence Lab Dr. Agustín Gravano / Dr. Diego Fernandez Slezak
\begin{itemize}
\item Tesis de grado Arneodo.
\item Otros investigadores: Alejandro Sabatini
\end{itemize}
\item INAUT, Instituto Nacional de Automática, San Juan, / Dr. Carlos Soria, Dr. Eugenio Orosco
\begin{itemize}
\item BCI Robótica (BCI híbridos, robótica asistiva)
\item Trabajan con Teodiano Freire Bastos en Brasil
\item \url{www.ncbi.nlm.nih.gov}
\item Otros investigadores: Mst. Ing. Fernando Auat Cheeín, Email: fauat@inaut.unsj.edu.ar
\end{itemize}
\item Instituto Argentino de Matemáticas Alberto Calderon / Bioing. Sergio Liberczuk, Dr. Bruno Cernuschi Frías
\begin{itemize}
\item Matemáticas y modelado del problema inverso.
\end{itemize}
\item ITBA / CiC del Dr. Juan Miguel Santos, \url{http://www.itba.edu.ar/es/id/centros/cic-centro-de-inteligencia-computacional}
\begin{itemize}
\item Proyecto Doctorado Robótica Asistiva BCI, Ing. Rodrigo Ramele
\item \url{http://www.unsam.edu.ar/tss/controlar-maquinas-con-el-pensamiento/978-3-319-13117-7_142}
\end{itemize}
\item UNC, Universidad Nacional de Cordoba
\begin{itemize}
\item Trabajo Final de Ingeniería: \url{http://www.electronicosonline.com/2013/07/08/crean-jovenes-argentinos-interface-cerebral-para-discapacitados/}
\item Carrera de Ingeniería Biomédica: Ing. Diego Beltramone
\end{itemize}
\item UNLP, LEICI / Dr. Ing. Enrique Spinelli (\url{http://www.ing.unlp.edu.ar/leici/esp/pspinelli.html})
\begin{itemize}
\item Tesis de Grado de García Pablo: \url{http://sedici.unlp.edu.ar/handle/10915/3800631605}
\item Tesis de Maestría de Andrea Noelia Bermudez Cicchino 31605
\item Cesar Caiafa (trabajó con Cichocki) \url{http://ccaiafa.wixsite.com/cesar}
\end{itemize}
\item Universidad Nacional de Tucuman, Instituto Superior de Investigaciones Biológicas (INSIBIO)
\begin{itemize}
\item \url{www.lamein.org}
\item Investigación sobre alternativas de codificación neural de los sistemas sensoriales.
\item Investigadores responsables: Dr. Carmelo Felice, Mst. Ing. Fernando Farfán
\item Email: cfelice@herrera.unt.edu.ar, ffarfan@herrera.unt.edu.ar
\end{itemize}
\item Laboratorio de Investigación y Desarrollo en Nuevas Tecnologías (LIDeNTec) - ANSES
\begin{itemize}
\item Desarrollo de BCI
\item Investigadores responsables: Dr. Mario Mastriani
\item Email: mmastri@gmail.com
\end{itemize}
\item INECO: Neurociencia general con eventuales aplicaciones en BCI

\begin{itemize}
\item Eugenia Hesse 
\item Agustín Ibañez
\end{itemize}

\item Hospital del Cruce Florencio Varela / IBCN Lab /  Dra. Silvia Kochen
\begin{itemize}
\item  \url{http://www.ibcn.fmed.uba.ar/200_grupos-lab-epilepsia-kochen.html}
\end{itemize}

\item Instituto Ferrero de Neurología y Sueño. Fundación Argentina de Estudio del Cerebro.
\begin{itemize}
\item Publicaron un excelente y clásico libro en castellano de \textit{Análisis Computado de EEG}.
\end{itemize}

\end{itemize}